% Options for packages loaded elsewhere
\PassOptionsToPackage{unicode}{hyperref}
\PassOptionsToPackage{hyphens}{url}
%
\documentclass[
  12pt,
]{article}
\usepackage{amsmath,amssymb}
\usepackage{iftex}
\ifPDFTeX
  \usepackage[T1]{fontenc}
  \usepackage[utf8]{inputenc}
  \usepackage{textcomp} % provide euro and other symbols
\else % if luatex or xetex
  \usepackage{unicode-math} % this also loads fontspec
  \defaultfontfeatures{Scale=MatchLowercase}
  \defaultfontfeatures[\rmfamily]{Ligatures=TeX,Scale=1}
\fi
\usepackage{lmodern}
\ifPDFTeX\else
  % xetex/luatex font selection
  \setmainfont[]{Times New Roman}
\fi
% Use upquote if available, for straight quotes in verbatim environments
\IfFileExists{upquote.sty}{\usepackage{upquote}}{}
\IfFileExists{microtype.sty}{% use microtype if available
  \usepackage[]{microtype}
  \UseMicrotypeSet[protrusion]{basicmath} % disable protrusion for tt fonts
}{}
\makeatletter
\@ifundefined{KOMAClassName}{% if non-KOMA class
  \IfFileExists{parskip.sty}{%
    \usepackage{parskip}
  }{% else
    \setlength{\parindent}{0pt}
    \setlength{\parskip}{6pt plus 2pt minus 1pt}}
}{% if KOMA class
  \KOMAoptions{parskip=half}}
\makeatother
\usepackage{xcolor}
\usepackage[margin=1in]{geometry}
\usepackage{color}
\usepackage{fancyvrb}
\newcommand{\VerbBar}{|}
\newcommand{\VERB}{\Verb[commandchars=\\\{\}]}
\DefineVerbatimEnvironment{Highlighting}{Verbatim}{commandchars=\\\{\}}
% Add ',fontsize=\small' for more characters per line
\usepackage{framed}
\definecolor{shadecolor}{RGB}{248,248,248}
\newenvironment{Shaded}{\begin{snugshade}}{\end{snugshade}}
\newcommand{\AlertTok}[1]{\textcolor[rgb]{0.94,0.16,0.16}{#1}}
\newcommand{\AnnotationTok}[1]{\textcolor[rgb]{0.56,0.35,0.01}{\textbf{\textit{#1}}}}
\newcommand{\AttributeTok}[1]{\textcolor[rgb]{0.13,0.29,0.53}{#1}}
\newcommand{\BaseNTok}[1]{\textcolor[rgb]{0.00,0.00,0.81}{#1}}
\newcommand{\BuiltInTok}[1]{#1}
\newcommand{\CharTok}[1]{\textcolor[rgb]{0.31,0.60,0.02}{#1}}
\newcommand{\CommentTok}[1]{\textcolor[rgb]{0.56,0.35,0.01}{\textit{#1}}}
\newcommand{\CommentVarTok}[1]{\textcolor[rgb]{0.56,0.35,0.01}{\textbf{\textit{#1}}}}
\newcommand{\ConstantTok}[1]{\textcolor[rgb]{0.56,0.35,0.01}{#1}}
\newcommand{\ControlFlowTok}[1]{\textcolor[rgb]{0.13,0.29,0.53}{\textbf{#1}}}
\newcommand{\DataTypeTok}[1]{\textcolor[rgb]{0.13,0.29,0.53}{#1}}
\newcommand{\DecValTok}[1]{\textcolor[rgb]{0.00,0.00,0.81}{#1}}
\newcommand{\DocumentationTok}[1]{\textcolor[rgb]{0.56,0.35,0.01}{\textbf{\textit{#1}}}}
\newcommand{\ErrorTok}[1]{\textcolor[rgb]{0.64,0.00,0.00}{\textbf{#1}}}
\newcommand{\ExtensionTok}[1]{#1}
\newcommand{\FloatTok}[1]{\textcolor[rgb]{0.00,0.00,0.81}{#1}}
\newcommand{\FunctionTok}[1]{\textcolor[rgb]{0.13,0.29,0.53}{\textbf{#1}}}
\newcommand{\ImportTok}[1]{#1}
\newcommand{\InformationTok}[1]{\textcolor[rgb]{0.56,0.35,0.01}{\textbf{\textit{#1}}}}
\newcommand{\KeywordTok}[1]{\textcolor[rgb]{0.13,0.29,0.53}{\textbf{#1}}}
\newcommand{\NormalTok}[1]{#1}
\newcommand{\OperatorTok}[1]{\textcolor[rgb]{0.81,0.36,0.00}{\textbf{#1}}}
\newcommand{\OtherTok}[1]{\textcolor[rgb]{0.56,0.35,0.01}{#1}}
\newcommand{\PreprocessorTok}[1]{\textcolor[rgb]{0.56,0.35,0.01}{\textit{#1}}}
\newcommand{\RegionMarkerTok}[1]{#1}
\newcommand{\SpecialCharTok}[1]{\textcolor[rgb]{0.81,0.36,0.00}{\textbf{#1}}}
\newcommand{\SpecialStringTok}[1]{\textcolor[rgb]{0.31,0.60,0.02}{#1}}
\newcommand{\StringTok}[1]{\textcolor[rgb]{0.31,0.60,0.02}{#1}}
\newcommand{\VariableTok}[1]{\textcolor[rgb]{0.00,0.00,0.00}{#1}}
\newcommand{\VerbatimStringTok}[1]{\textcolor[rgb]{0.31,0.60,0.02}{#1}}
\newcommand{\WarningTok}[1]{\textcolor[rgb]{0.56,0.35,0.01}{\textbf{\textit{#1}}}}
\usepackage{longtable,booktabs,array}
\usepackage{calc} % for calculating minipage widths
% Correct order of tables after \paragraph or \subparagraph
\usepackage{etoolbox}
\makeatletter
\patchcmd\longtable{\par}{\if@noskipsec\mbox{}\fi\par}{}{}
\makeatother
% Allow footnotes in longtable head/foot
\IfFileExists{footnotehyper.sty}{\usepackage{footnotehyper}}{\usepackage{footnote}}
\makesavenoteenv{longtable}
\usepackage{graphicx}
\makeatletter
\def\maxwidth{\ifdim\Gin@nat@width>\linewidth\linewidth\else\Gin@nat@width\fi}
\def\maxheight{\ifdim\Gin@nat@height>\textheight\textheight\else\Gin@nat@height\fi}
\makeatother
% Scale images if necessary, so that they will not overflow the page
% margins by default, and it is still possible to overwrite the defaults
% using explicit options in \includegraphics[width, height, ...]{}
\setkeys{Gin}{width=\maxwidth,height=\maxheight,keepaspectratio}
% Set default figure placement to htbp
\makeatletter
\def\fps@figure{htbp}
\makeatother
\setlength{\emergencystretch}{3em} % prevent overfull lines
\providecommand{\tightlist}{%
  \setlength{\itemsep}{0pt}\setlength{\parskip}{0pt}}
\setcounter{secnumdepth}{5}
\usepackage{tcolorbox}
\usepackage{amssymb}
\usepackage{yfonts}
\usepackage{bm}

\newtcolorbox{greybox}{
  colback=white,
  colframe=blue,
  coltext=black,
  boxsep=5pt,
  arc=4pt}
  
\newcommand{\ds}[4]{\sum_{{#1}=1}^{#3}\sum_{{#2}=1}^{#4}}
\newcommand{\us}[3]{\mathop{\sum\sum}_{1\leq{#2}<{#1}\leq{#3}}}

\newcommand{\ol}[1]{\overline{#1}}
\newcommand{\ul}[1]{\underline{#1}}

\newcommand{\amin}[1]{\mathop{\text{argmin}}_{#1}}
\newcommand{\amax}[1]{\mathop{\text{argmax}}_{#1}}

\newcommand{\ci}{\perp\!\!\!\perp}

\newcommand{\mc}[1]{\mathcal{#1}}
\newcommand{\mb}[1]{\mathbb{#1}}
\newcommand{\mf}[1]{\mathfrak{#1}}

\newcommand{\eps}{\epsilon}
\newcommand{\lbd}{\lambda}
\newcommand{\alp}{\alpha}
\newcommand{\df}{=:}
\newcommand{\am}[1]{\mathop{\text{argmin}}_{#1}}
\newcommand{\ls}[2]{\mathop{\sum\sum}_{#1}^{#2}}
\newcommand{\ijs}{\mathop{\sum\sum}_{1\leq i<j\leq n}}
\newcommand{\jis}{\mathop{\sum\sum}_{1\leq j<i\leq n}}
\newcommand{\sij}{\sum_{i=1}^n\sum_{j=1}^n}
	
\ifLuaTeX
  \usepackage{selnolig}  % disable illegal ligatures
\fi
\IfFileExists{bookmark.sty}{\usepackage{bookmark}}{\usepackage{hyperref}}
\IfFileExists{xurl.sty}{\usepackage{xurl}}{} % add URL line breaks if available
\urlstyle{same}
\hypersetup{
  pdftitle={Organizing Multidimensional Scaling I/O and Computation},
  pdfauthor={Jan de Leeuw - University of California Los Angeles},
  hidelinks,
  pdfcreator={LaTeX via pandoc}}

\title{Organizing Multidimensional Scaling I/O and Computation}
\author{Jan de Leeuw - University of California Los Angeles}
\date{Started October 10 2023, Version of October 10, 2023}

\begin{document}
\maketitle
\begin{abstract}
We develop data structures suitable for multidimensional scaling in the smacof framework. Code in both R and C is included.
\end{abstract}

{
\setcounter{tocdepth}{4}
\tableofcontents
}
\textbf{Note:} This is a working paper which will be expanded/updated frequently. All suggestions for improvement are welcome. All Rmd, tex, html, pdf, R, and C files are in the public domain. Attribution
will be appreciated, but is not required. The files can be found at
\url{https://github.com/deleeuw/mdsStruct}.

\section{Introduction}\label{introduction}

The loss function in (metric, least squares, Euclidean, symmetric) Multidimensional Scaling (MDS)
is
\[
\sigma(X):=\frac12\jis w_{ij}(\delta_{ij}-d_{ij}(X))^2.
\]

This assume symmetry and it uses all elements below the diagonal of both \(W\), \(\Delta\), and \(D(X)\).
For missing data we set \(w_{ij}=0\).

\section{Example}\label{example}

Here is a small input example.

\begin{Shaded}
\begin{Highlighting}[]
\NormalTok{n }\OtherTok{\textless{}{-}} \DecValTok{5}
\NormalTok{row }\OtherTok{\textless{}{-}}\NormalTok{ col }\OtherTok{\textless{}{-}} \ConstantTok{NULL}
\FunctionTok{set.seed}\NormalTok{(}\DecValTok{12345}\NormalTok{)}
\NormalTok{delta }\OtherTok{\textless{}{-}} \FunctionTok{sample}\NormalTok{(}\DecValTok{1}\SpecialCharTok{:}\DecValTok{5}\NormalTok{, }\DecValTok{10}\NormalTok{, }\AttributeTok{replace =} \ConstantTok{TRUE}\NormalTok{)}
\NormalTok{w }\OtherTok{\textless{}{-}} \FunctionTok{sample}\NormalTok{(}\DecValTok{1}\SpecialCharTok{:}\DecValTok{5}\NormalTok{, }\DecValTok{10}\NormalTok{, }\AttributeTok{replace =} \ConstantTok{TRUE}\NormalTok{)}

\ControlFlowTok{for}\NormalTok{ (j }\ControlFlowTok{in} \DecValTok{1}\SpecialCharTok{:}\NormalTok{(n }\SpecialCharTok{{-}} \DecValTok{1}\NormalTok{)) \{}
  \ControlFlowTok{for}\NormalTok{ (i }\ControlFlowTok{in}\NormalTok{ (j }\SpecialCharTok{+} \DecValTok{1}\NormalTok{)}\SpecialCharTok{:}\NormalTok{n) \{}
\NormalTok{    row }\OtherTok{\textless{}{-}} \FunctionTok{c}\NormalTok{(row, i)}
\NormalTok{    col }\OtherTok{\textless{}{-}} \FunctionTok{c}\NormalTok{(col, j)}
\NormalTok{  \}}
\NormalTok{\}}
\FunctionTok{data.frame}\NormalTok{(}\AttributeTok{row =}\NormalTok{ row, }\AttributeTok{col =}\NormalTok{ col, }\AttributeTok{delta =}\NormalTok{ delta, }\AttributeTok{w =}\NormalTok{ w)}
\end{Highlighting}
\end{Shaded}

\begin{verbatim}
##    row col delta w
## 1    2   1     3 1
## 2    3   1     2 4
## 3    4   1     4 4
## 4    5   1     2 2
## 5    3   2     5 4
## 6    4   2     3 3
## 7    5   2     2 1
## 8    4   3     3 5
## 9    5   3     2 4
## 10   5   4     1 2
\end{verbatim}

We use this example as input for the R version of \emph{smacofSort()}, which results in the
\emph{smacofStructure}

\begin{Shaded}
\begin{Highlighting}[]
\FunctionTok{dyn.load}\NormalTok{(}\StringTok{"smacofSort.so"}\NormalTok{)}
\FunctionTok{smacofSortR}\NormalTok{(delta, w, row, col)}
\end{Highlighting}
\end{Shaded}

\begin{verbatim}
##    row col delta w block dist dhat
## 1    5   4     1 2     1    0    0
## 2    5   3     2 4     2    0    0
## 3    5   1     2 2     2    0    0
## 4    3   1     2 4     2    0    0
## 5    5   2     2 1     2    0    0
## 6    4   2     3 3     3    0    0
## 7    4   3     3 5     3    0    0
## 8    2   1     3 1     3    0    0
## 9    4   1     4 4     4    0    0
## 10   3   2     5 4     5    0    0
\end{verbatim}

\begin{Shaded}
\begin{Highlighting}[]
\FunctionTok{dyn.unload}\NormalTok{(}\StringTok{"smacofSort.so"}\NormalTok{)}
\end{Highlighting}
\end{Shaded}

The \emph{dist} element in the data frame is zero, because we have not computed distances yet.
Given a configuration \(X\) the \emph{row} and \emph{col} elements in the data frame allow from
straightforward computation of distances. In the case of nonmetric MDS, or more generally
in MDS with transformed dissimilarities, the \emph{dhat} column will be filled as well.
The fact that preprocessing with \emph{smacofSort()} gives the ordered dissimilarities, as
well as the tie blocks, is especially useful in the ordinal case, both with monotone
polynomials and monotone splines. In the metric (ratio) case there is no need for the
\emph{dhat} column.

It is obvious how this \emph{smacofStructure} can be adapted if there are multiway data.If we have row-conditional or matrix-conditional data, then we use one of these smacofStructures for each
row or each matrix.

\section{Appendix: Code}\label{appendix-code}

We use \emph{qsort} to sort the rows of the input data frame by increasing delta. The sorting
is done in C, the R version is a wrapper around the compiled C code. The C code also
contains a main which analyzes the same small example as we have used in the text.
Compile with ``clang -o runner -O2 smacofSort.c'' and then start ``runner'' in the shell.
The C code uses the .C() interface in R, which can be improved using .Call(),
but probably in this case with little gain.

\subsection{smacofSort.R}\label{smacofsort.r}

\begin{Shaded}
\begin{Highlighting}[]
\NormalTok{smacofSortR }\OtherTok{\textless{}{-}} \ControlFlowTok{function}\NormalTok{(delta, w, row, col, }\AttributeTok{eps =} \FloatTok{1e{-}6}\NormalTok{) \{}
\NormalTok{  n }\OtherTok{\textless{}{-}} \FunctionTok{length}\NormalTok{(delta)}
\NormalTok{  h }\OtherTok{\textless{}{-}} \FunctionTok{.C}\NormalTok{(}
    \StringTok{"smacofSort"}\NormalTok{,}
    \AttributeTok{delta =} \FunctionTok{as.double}\NormalTok{(delta),}
    \AttributeTok{weight =} \FunctionTok{as.double}\NormalTok{(w),}
    \AttributeTok{row =} \FunctionTok{as.integer}\NormalTok{(row),}
    \AttributeTok{col =} \FunctionTok{as.integer}\NormalTok{(col),}
    \AttributeTok{index =} \FunctionTok{as.integer}\NormalTok{(}\DecValTok{0}\SpecialCharTok{:}\NormalTok{(n }\SpecialCharTok{{-}} \DecValTok{1}\NormalTok{)),}
    \AttributeTok{n =} \FunctionTok{as.integer}\NormalTok{(n)}
\NormalTok{  )}
\NormalTok{  g }\OtherTok{\textless{}{-}} \FunctionTok{.C}\NormalTok{(}
    \StringTok{"smacofTieBlocks"}\NormalTok{,}
    \AttributeTok{delta =} \FunctionTok{as.double}\NormalTok{(h}\SpecialCharTok{$}\NormalTok{delta),}
    \AttributeTok{block =} \FunctionTok{as.integer}\NormalTok{(}\FunctionTok{rep}\NormalTok{(}\DecValTok{0}\NormalTok{, n)),}
    \AttributeTok{eps =} \FunctionTok{as.double}\NormalTok{(eps),}
    \AttributeTok{n =} \FunctionTok{as.integer}\NormalTok{(n)}
\NormalTok{  )}
  \FunctionTok{return}\NormalTok{(}\FunctionTok{data.frame}\NormalTok{(}\AttributeTok{row =}\NormalTok{ h}\SpecialCharTok{$}\NormalTok{row, }\AttributeTok{col =}\NormalTok{ h}\SpecialCharTok{$}\NormalTok{col, }\AttributeTok{delta =}\NormalTok{ h}\SpecialCharTok{$}\NormalTok{delta, }\AttributeTok{w =}\NormalTok{ h}\SpecialCharTok{$}\NormalTok{w, }\AttributeTok{block =}\NormalTok{ g}\SpecialCharTok{$}\NormalTok{block, }\AttributeTok{dist =} \DecValTok{0}\NormalTok{, }\AttributeTok{dhat =} \DecValTok{0}\NormalTok{))}
\NormalTok{\}}
\end{Highlighting}
\end{Shaded}

\subsection{smacofSort.c}\label{smacofsort.c}

\begin{Shaded}
\begin{Highlighting}[]

\PreprocessorTok{\#include }\ImportTok{\textless{}math.h\textgreater{}}
\PreprocessorTok{\#include }\ImportTok{\textless{}stdio.h\textgreater{}}
\PreprocessorTok{\#include }\ImportTok{\textless{}stdlib.h\textgreater{}}

\DataTypeTok{int}\NormalTok{ smacofComparison}\OperatorTok{(}\DataTypeTok{const} \DataTypeTok{void} \OperatorTok{*}\NormalTok{px}\OperatorTok{,} \DataTypeTok{const} \DataTypeTok{void} \OperatorTok{*}\NormalTok{py}\OperatorTok{);}
\DataTypeTok{void}\NormalTok{ smacofSort}\OperatorTok{(}\DataTypeTok{double} \OperatorTok{*}\NormalTok{delta}\OperatorTok{,} \DataTypeTok{double} \OperatorTok{*}\NormalTok{weight}\OperatorTok{,} \DataTypeTok{int} \OperatorTok{*}\NormalTok{row}\OperatorTok{,} \DataTypeTok{int} \OperatorTok{*}\NormalTok{col}\OperatorTok{,} \DataTypeTok{int} \OperatorTok{*}\NormalTok{index}\OperatorTok{,}
                \DataTypeTok{const} \DataTypeTok{int} \OperatorTok{*}\NormalTok{n}\OperatorTok{);}
\DataTypeTok{void}\NormalTok{ smacofTieBlocks}\OperatorTok{(}\DataTypeTok{const} \DataTypeTok{double} \OperatorTok{*}\NormalTok{x}\OperatorTok{,} \DataTypeTok{int} \OperatorTok{*}\NormalTok{it}\OperatorTok{,} \DataTypeTok{double} \OperatorTok{*}\NormalTok{eps}\OperatorTok{,} \DataTypeTok{const} \DataTypeTok{int} \OperatorTok{*}\NormalTok{n}\OperatorTok{);}

\KeywordTok{struct}\NormalTok{ smacofData }\OperatorTok{\{}
    \DataTypeTok{int}\NormalTok{ index}\OperatorTok{;}
    \DataTypeTok{int}\NormalTok{ row}\OperatorTok{;}
    \DataTypeTok{int}\NormalTok{ col}\OperatorTok{;}
    \DataTypeTok{double}\NormalTok{ delta}\OperatorTok{;}
    \DataTypeTok{double}\NormalTok{ weight}\OperatorTok{;}
\OperatorTok{\};}

\DataTypeTok{int}\NormalTok{ smacofComparison}\OperatorTok{(}\DataTypeTok{const} \DataTypeTok{void} \OperatorTok{*}\NormalTok{px}\OperatorTok{,} \DataTypeTok{const} \DataTypeTok{void} \OperatorTok{*}\NormalTok{py}\OperatorTok{)} \OperatorTok{\{}
    \DataTypeTok{double}\NormalTok{ x }\OperatorTok{=} \OperatorTok{((}\KeywordTok{struct}\NormalTok{ smacofData }\OperatorTok{*)}\NormalTok{px}\OperatorTok{){-}\textgreater{}}\NormalTok{delta}\OperatorTok{;}
    \DataTypeTok{double}\NormalTok{ y }\OperatorTok{=} \OperatorTok{((}\KeywordTok{struct}\NormalTok{ smacofData }\OperatorTok{*)}\NormalTok{py}\OperatorTok{){-}\textgreater{}}\NormalTok{delta}\OperatorTok{;}
    \ControlFlowTok{return} \OperatorTok{(}\DataTypeTok{int}\OperatorTok{)}\NormalTok{copysign}\OperatorTok{(}\FloatTok{1.0}\OperatorTok{,}\NormalTok{ x }\OperatorTok{{-}}\NormalTok{ y}\OperatorTok{);}
\OperatorTok{\}}

\DataTypeTok{void}\NormalTok{ smacofSort}\OperatorTok{(}\DataTypeTok{double} \OperatorTok{*}\NormalTok{delta}\OperatorTok{,} \DataTypeTok{double} \OperatorTok{*}\NormalTok{weight}\OperatorTok{,} \DataTypeTok{int} \OperatorTok{*}\NormalTok{row}\OperatorTok{,} \DataTypeTok{int} \OperatorTok{*}\NormalTok{col}\OperatorTok{,} \DataTypeTok{int} \OperatorTok{*}\NormalTok{index}\OperatorTok{,}
                \DataTypeTok{const} \DataTypeTok{int} \OperatorTok{*}\NormalTok{n}\OperatorTok{)} \OperatorTok{\{}
    \DataTypeTok{int}\NormalTok{ nn }\OperatorTok{=} \OperatorTok{*}\NormalTok{n}\OperatorTok{;}
    \KeywordTok{struct}\NormalTok{ smacofData }\OperatorTok{*}\NormalTok{xi }\OperatorTok{=} \OperatorTok{(}\KeywordTok{struct}\NormalTok{ smacofData }\OperatorTok{*)}\NormalTok{calloc}\OperatorTok{(}
        \OperatorTok{(}\DataTypeTok{size\_t}\OperatorTok{)}\NormalTok{nn}\OperatorTok{,} \OperatorTok{(}\DataTypeTok{size\_t}\OperatorTok{)}\KeywordTok{sizeof}\OperatorTok{(}\KeywordTok{struct}\NormalTok{ smacofData}\OperatorTok{));}
    \ControlFlowTok{for} \OperatorTok{(}\DataTypeTok{int}\NormalTok{ i }\OperatorTok{=} \DecValTok{0}\OperatorTok{;}\NormalTok{ i }\OperatorTok{\textless{}}\NormalTok{ nn}\OperatorTok{;}\NormalTok{ i}\OperatorTok{++)} \OperatorTok{\{}
\NormalTok{        xi}\OperatorTok{[}\NormalTok{i}\OperatorTok{].}\NormalTok{index }\OperatorTok{=}\NormalTok{ i}\OperatorTok{;}
\NormalTok{        xi}\OperatorTok{[}\NormalTok{i}\OperatorTok{].}\NormalTok{row }\OperatorTok{=}\NormalTok{ row}\OperatorTok{[}\NormalTok{i}\OperatorTok{];}
\NormalTok{        xi}\OperatorTok{[}\NormalTok{i}\OperatorTok{].}\NormalTok{col }\OperatorTok{=}\NormalTok{ col}\OperatorTok{[}\NormalTok{i}\OperatorTok{];}
\NormalTok{        xi}\OperatorTok{[}\NormalTok{i}\OperatorTok{].}\NormalTok{delta }\OperatorTok{=}\NormalTok{ delta}\OperatorTok{[}\NormalTok{i}\OperatorTok{];}
\NormalTok{        xi}\OperatorTok{[}\NormalTok{i}\OperatorTok{].}\NormalTok{weight }\OperatorTok{=}\NormalTok{ weight}\OperatorTok{[}\NormalTok{i}\OperatorTok{];}
    \OperatorTok{\}}
    \OperatorTok{(}\DataTypeTok{void}\OperatorTok{)}\NormalTok{qsort}\OperatorTok{(}\NormalTok{xi}\OperatorTok{,} \OperatorTok{(}\DataTypeTok{size\_t}\OperatorTok{)}\NormalTok{nn}\OperatorTok{,} \OperatorTok{(}\DataTypeTok{size\_t}\OperatorTok{)}\KeywordTok{sizeof}\OperatorTok{(}\KeywordTok{struct}\NormalTok{ smacofData}\OperatorTok{),}
\NormalTok{                smacofComparison}\OperatorTok{);}
    \ControlFlowTok{for} \OperatorTok{(}\DataTypeTok{int}\NormalTok{ i }\OperatorTok{=} \DecValTok{0}\OperatorTok{;}\NormalTok{ i }\OperatorTok{\textless{}}\NormalTok{ nn}\OperatorTok{;}\NormalTok{ i}\OperatorTok{++)} \OperatorTok{\{}
\NormalTok{        index}\OperatorTok{[}\NormalTok{i}\OperatorTok{]} \OperatorTok{=}\NormalTok{ xi}\OperatorTok{[}\NormalTok{i}\OperatorTok{].}\NormalTok{index}\OperatorTok{;}
\NormalTok{        row}\OperatorTok{[}\NormalTok{i}\OperatorTok{]} \OperatorTok{=}\NormalTok{ xi}\OperatorTok{[}\NormalTok{i}\OperatorTok{].}\NormalTok{row}\OperatorTok{;}
\NormalTok{        col}\OperatorTok{[}\NormalTok{i}\OperatorTok{]} \OperatorTok{=}\NormalTok{ xi}\OperatorTok{[}\NormalTok{i}\OperatorTok{].}\NormalTok{col}\OperatorTok{;}
\NormalTok{        delta}\OperatorTok{[}\NormalTok{i}\OperatorTok{]} \OperatorTok{=}\NormalTok{ xi}\OperatorTok{[}\NormalTok{i}\OperatorTok{].}\NormalTok{delta}\OperatorTok{;}
\NormalTok{        weight}\OperatorTok{[}\NormalTok{i}\OperatorTok{]} \OperatorTok{=}\NormalTok{ xi}\OperatorTok{[}\NormalTok{i}\OperatorTok{].}\NormalTok{weight}\OperatorTok{;}
    \OperatorTok{\}}
\NormalTok{    free}\OperatorTok{(}\NormalTok{xi}\OperatorTok{);}
    \ControlFlowTok{return}\OperatorTok{;}
\OperatorTok{\}}

\DataTypeTok{void}\NormalTok{ smacofTieBlocks}\OperatorTok{(}\DataTypeTok{const} \DataTypeTok{double} \OperatorTok{*}\NormalTok{delta}\OperatorTok{,} \DataTypeTok{int} \OperatorTok{*}\NormalTok{block}\OperatorTok{,} \DataTypeTok{double} \OperatorTok{*}\NormalTok{eps}\OperatorTok{,}
                     \DataTypeTok{const} \DataTypeTok{int} \OperatorTok{*}\NormalTok{n}\OperatorTok{)} \OperatorTok{\{}
    \DataTypeTok{int}\NormalTok{ nn }\OperatorTok{=} \OperatorTok{*}\NormalTok{n}\OperatorTok{;}
\NormalTok{    block}\OperatorTok{[}\DecValTok{0}\OperatorTok{]} \OperatorTok{=} \DecValTok{1}\OperatorTok{;}
    \ControlFlowTok{for} \OperatorTok{(}\DataTypeTok{int}\NormalTok{ i }\OperatorTok{=} \DecValTok{1}\OperatorTok{;}\NormalTok{ i }\OperatorTok{\textless{}}\NormalTok{ nn}\OperatorTok{;}\NormalTok{ i}\OperatorTok{++)} \OperatorTok{\{}
        \ControlFlowTok{if} \OperatorTok{(}\NormalTok{fabs}\OperatorTok{(}\NormalTok{delta}\OperatorTok{[}\NormalTok{i}\OperatorTok{]} \OperatorTok{{-}}\NormalTok{ delta}\OperatorTok{[}\NormalTok{i }\OperatorTok{{-}} \DecValTok{1}\OperatorTok{])} \OperatorTok{\textless{}} \OperatorTok{*}\NormalTok{eps}\OperatorTok{)} \OperatorTok{\{}
\NormalTok{            block}\OperatorTok{[}\NormalTok{i}\OperatorTok{]} \OperatorTok{=}\NormalTok{ block}\OperatorTok{[}\NormalTok{i }\OperatorTok{{-}} \DecValTok{1}\OperatorTok{];}
        \OperatorTok{\}} \ControlFlowTok{else} \OperatorTok{\{}
\NormalTok{            block}\OperatorTok{[}\NormalTok{i}\OperatorTok{]} \OperatorTok{=}\NormalTok{ block}\OperatorTok{[}\NormalTok{i }\OperatorTok{{-}} \DecValTok{1}\OperatorTok{]} \OperatorTok{+} \DecValTok{1}\OperatorTok{;}
        \OperatorTok{\}}
    \OperatorTok{\}}
    \ControlFlowTok{return}\OperatorTok{;}
\OperatorTok{\}}

\DataTypeTok{int}\NormalTok{ main}\OperatorTok{(}\DataTypeTok{void}\OperatorTok{)} \OperatorTok{\{}
    \DataTypeTok{int}\NormalTok{ index}\OperatorTok{[}\DecValTok{10}\OperatorTok{]} \OperatorTok{=} \OperatorTok{\{}\DecValTok{0}\OperatorTok{,} \DecValTok{1}\OperatorTok{,} \DecValTok{2}\OperatorTok{,} \DecValTok{3}\OperatorTok{,} \DecValTok{4}\OperatorTok{,} \DecValTok{5}\OperatorTok{,} \DecValTok{6}\OperatorTok{,} \DecValTok{7}\OperatorTok{,} \DecValTok{8}\OperatorTok{,} \DecValTok{9}\OperatorTok{\};}
    \DataTypeTok{double}\NormalTok{ delta}\OperatorTok{[}\DecValTok{10}\OperatorTok{]} \OperatorTok{=} \OperatorTok{\{}\FloatTok{3.0}\OperatorTok{,} \FloatTok{2.0}\OperatorTok{,} \FloatTok{4.0}\OperatorTok{,} \FloatTok{2.0}\OperatorTok{,} \FloatTok{5.0}\OperatorTok{,} \FloatTok{3.0}\OperatorTok{,} \FloatTok{2.0}\OperatorTok{,} \FloatTok{3.0}\OperatorTok{,} \FloatTok{2.0}\OperatorTok{,} \FloatTok{1.0}\OperatorTok{\};}
    \DataTypeTok{double}\NormalTok{ weight}\OperatorTok{[}\DecValTok{10}\OperatorTok{]} \OperatorTok{=} \OperatorTok{\{}\FloatTok{1.0}\OperatorTok{,} \FloatTok{4.0}\OperatorTok{,} \FloatTok{4.0}\OperatorTok{,} \FloatTok{2.0}\OperatorTok{,} \FloatTok{4.0}\OperatorTok{,} \FloatTok{3.0}\OperatorTok{,} \FloatTok{1.0}\OperatorTok{,} \FloatTok{5.0}\OperatorTok{,} \FloatTok{4.0}\OperatorTok{,} \FloatTok{2.0}\OperatorTok{\};}
    \DataTypeTok{double}\NormalTok{ dist}\OperatorTok{[}\DecValTok{10}\OperatorTok{]} \OperatorTok{=} \OperatorTok{\{}\FloatTok{0.0}\OperatorTok{,} \FloatTok{0.0}\OperatorTok{,} \FloatTok{0.0}\OperatorTok{,} \FloatTok{0.0}\OperatorTok{,} \FloatTok{0.0}\OperatorTok{,} \FloatTok{0.0}\OperatorTok{,} \FloatTok{0.0}\OperatorTok{,} \FloatTok{0.0}\OperatorTok{,} \FloatTok{0.0}\OperatorTok{,} \FloatTok{0.0}\OperatorTok{\};}
    \DataTypeTok{double}\NormalTok{ dhat}\OperatorTok{[}\DecValTok{10}\OperatorTok{]} \OperatorTok{=} \OperatorTok{\{}\FloatTok{0.0}\OperatorTok{,} \FloatTok{0.0}\OperatorTok{,} \FloatTok{0.0}\OperatorTok{,} \FloatTok{0.0}\OperatorTok{,} \FloatTok{0.0}\OperatorTok{,} \FloatTok{0.0}\OperatorTok{,} \FloatTok{0.0}\OperatorTok{,} \FloatTok{0.0}\OperatorTok{,} \FloatTok{0.0}\OperatorTok{,} \FloatTok{0.0}\OperatorTok{\};}
    \DataTypeTok{int}\NormalTok{ row}\OperatorTok{[}\DecValTok{10}\OperatorTok{]} \OperatorTok{=} \OperatorTok{\{}\DecValTok{2}\OperatorTok{,} \DecValTok{3}\OperatorTok{,} \DecValTok{4}\OperatorTok{,} \DecValTok{5}\OperatorTok{,} \DecValTok{3}\OperatorTok{,} \DecValTok{4}\OperatorTok{,} \DecValTok{5}\OperatorTok{,} \DecValTok{4}\OperatorTok{,} \DecValTok{5}\OperatorTok{,} \DecValTok{5}\OperatorTok{\};}
    \DataTypeTok{int}\NormalTok{ col}\OperatorTok{[}\DecValTok{10}\OperatorTok{]} \OperatorTok{=} \OperatorTok{\{}\DecValTok{1}\OperatorTok{,} \DecValTok{1}\OperatorTok{,} \DecValTok{1}\OperatorTok{,} \DecValTok{1}\OperatorTok{,} \DecValTok{2}\OperatorTok{,} \DecValTok{2}\OperatorTok{,} \DecValTok{2}\OperatorTok{,} \DecValTok{3}\OperatorTok{,} \DecValTok{3}\OperatorTok{,} \DecValTok{4}\OperatorTok{\};}
    \DataTypeTok{int}\NormalTok{ block}\OperatorTok{[}\DecValTok{10}\OperatorTok{]} \OperatorTok{=} \OperatorTok{\{}\DecValTok{0}\OperatorTok{,} \DecValTok{0}\OperatorTok{,} \DecValTok{0}\OperatorTok{,} \DecValTok{0}\OperatorTok{,} \DecValTok{0}\OperatorTok{,} \DecValTok{0}\OperatorTok{,} \DecValTok{0}\OperatorTok{,} \DecValTok{0}\OperatorTok{,} \DecValTok{0}\OperatorTok{,} \DecValTok{0}\OperatorTok{\};}
    \DataTypeTok{int}\NormalTok{ n }\OperatorTok{=} \DecValTok{10}\OperatorTok{;}
    \DataTypeTok{double}\NormalTok{ eps }\OperatorTok{=} \FloatTok{1e{-}6}\OperatorTok{;}
    \ControlFlowTok{for} \OperatorTok{(}\DataTypeTok{int}\NormalTok{ i }\OperatorTok{=} \DecValTok{0}\OperatorTok{;}\NormalTok{ i }\OperatorTok{\textless{}}\NormalTok{ n}\OperatorTok{;}\NormalTok{ i}\OperatorTok{++)} \OperatorTok{\{}
\NormalTok{        printf}\OperatorTok{(}
            \StringTok{"row}\SpecialCharTok{\%2d}\StringTok{ col}\SpecialCharTok{\%2d}\StringTok{ delta }\SpecialCharTok{\%4.2f}\StringTok{ weight }\SpecialCharTok{\%4.2f}\StringTok{ block}\SpecialCharTok{\%2d}\StringTok{ dist }\SpecialCharTok{\%4.2f}\StringTok{ dhat "}
            \StringTok{"}\SpecialCharTok{\%4.2f\textbackslash{}n}\StringTok{"}\OperatorTok{,}
\NormalTok{            row}\OperatorTok{[}\NormalTok{i}\OperatorTok{],}\NormalTok{ col}\OperatorTok{[}\NormalTok{i}\OperatorTok{],}\NormalTok{ delta}\OperatorTok{[}\NormalTok{i}\OperatorTok{],}\NormalTok{ weight}\OperatorTok{[}\NormalTok{i}\OperatorTok{],}\NormalTok{ block}\OperatorTok{[}\NormalTok{i}\OperatorTok{],}\NormalTok{ dist}\OperatorTok{[}\NormalTok{i}\OperatorTok{],}\NormalTok{ dhat}\OperatorTok{[}\NormalTok{i}\OperatorTok{]);}
    \OperatorTok{\}}
\NormalTok{    printf}\OperatorTok{(}\StringTok{"}\SpecialCharTok{\textbackslash{}n\textbackslash{}n}\StringTok{"}\OperatorTok{);}
    \OperatorTok{(}\DataTypeTok{void}\OperatorTok{)}\NormalTok{smacofSort}\OperatorTok{(}\NormalTok{delta}\OperatorTok{,}\NormalTok{ weight}\OperatorTok{,}\NormalTok{ row}\OperatorTok{,}\NormalTok{ col}\OperatorTok{,}\NormalTok{ index}\OperatorTok{,} \OperatorTok{\&}\NormalTok{n}\OperatorTok{);}
    \OperatorTok{(}\DataTypeTok{void}\OperatorTok{)}\NormalTok{smacofTieBlocks}\OperatorTok{(}\NormalTok{delta}\OperatorTok{,}\NormalTok{ block}\OperatorTok{,} \OperatorTok{\&}\NormalTok{eps}\OperatorTok{,} \OperatorTok{\&}\NormalTok{n}\OperatorTok{);}
    \ControlFlowTok{for} \OperatorTok{(}\DataTypeTok{int}\NormalTok{ i }\OperatorTok{=} \DecValTok{0}\OperatorTok{;}\NormalTok{ i }\OperatorTok{\textless{}}\NormalTok{ n}\OperatorTok{;}\NormalTok{ i}\OperatorTok{++)} \OperatorTok{\{}
\NormalTok{        printf}\OperatorTok{(}
            \StringTok{"row}\SpecialCharTok{\%2d}\StringTok{ col}\SpecialCharTok{\%2d}\StringTok{ delta }\SpecialCharTok{\%4.2f}\StringTok{ weight }\SpecialCharTok{\%4.2f}\StringTok{ block}\SpecialCharTok{\%2d}\StringTok{ dist }\SpecialCharTok{\%4.2f}\StringTok{ dhat "}
            \StringTok{"}\SpecialCharTok{\%4.2f\textbackslash{}n}\StringTok{"}\OperatorTok{,}
\NormalTok{            row}\OperatorTok{[}\NormalTok{i}\OperatorTok{],}\NormalTok{ col}\OperatorTok{[}\NormalTok{i}\OperatorTok{],}\NormalTok{ delta}\OperatorTok{[}\NormalTok{i}\OperatorTok{],}\NormalTok{ weight}\OperatorTok{[}\NormalTok{i}\OperatorTok{],}\NormalTok{ block}\OperatorTok{[}\NormalTok{i}\OperatorTok{],}\NormalTok{ dist}\OperatorTok{[}\NormalTok{i}\OperatorTok{],}\NormalTok{ dhat}\OperatorTok{[}\NormalTok{i}\OperatorTok{]);}
    \OperatorTok{\}}
    \ControlFlowTok{return} \OperatorTok{(}\NormalTok{EXIT\_SUCCESS}\OperatorTok{);}
\OperatorTok{\}}
\end{Highlighting}
\end{Shaded}

\section{References}\label{references}

\end{document}

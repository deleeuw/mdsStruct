% Options for packages loaded elsewhere
\PassOptionsToPackage{unicode}{hyperref}
\PassOptionsToPackage{hyphens}{url}
%
\documentclass[
  12pt,
]{article}
\usepackage{amsmath,amssymb}
\usepackage{iftex}
\ifPDFTeX
  \usepackage[T1]{fontenc}
  \usepackage[utf8]{inputenc}
  \usepackage{textcomp} % provide euro and other symbols
\else % if luatex or xetex
  \usepackage{unicode-math} % this also loads fontspec
  \defaultfontfeatures{Scale=MatchLowercase}
  \defaultfontfeatures[\rmfamily]{Ligatures=TeX,Scale=1}
\fi
\usepackage{lmodern}
\ifPDFTeX\else
  % xetex/luatex font selection
  \setmainfont[]{Times New Roman}
\fi
% Use upquote if available, for straight quotes in verbatim environments
\IfFileExists{upquote.sty}{\usepackage{upquote}}{}
\IfFileExists{microtype.sty}{% use microtype if available
  \usepackage[]{microtype}
  \UseMicrotypeSet[protrusion]{basicmath} % disable protrusion for tt fonts
}{}
\makeatletter
\@ifundefined{KOMAClassName}{% if non-KOMA class
  \IfFileExists{parskip.sty}{%
    \usepackage{parskip}
  }{% else
    \setlength{\parindent}{0pt}
    \setlength{\parskip}{6pt plus 2pt minus 1pt}}
}{% if KOMA class
  \KOMAoptions{parskip=half}}
\makeatother
\usepackage{xcolor}
\usepackage[margin=1in]{geometry}
\usepackage{color}
\usepackage{fancyvrb}
\newcommand{\VerbBar}{|}
\newcommand{\VERB}{\Verb[commandchars=\\\{\}]}
\DefineVerbatimEnvironment{Highlighting}{Verbatim}{commandchars=\\\{\}}
% Add ',fontsize=\small' for more characters per line
\usepackage{framed}
\definecolor{shadecolor}{RGB}{248,248,248}
\newenvironment{Shaded}{\begin{snugshade}}{\end{snugshade}}
\newcommand{\AlertTok}[1]{\textcolor[rgb]{0.94,0.16,0.16}{#1}}
\newcommand{\AnnotationTok}[1]{\textcolor[rgb]{0.56,0.35,0.01}{\textbf{\textit{#1}}}}
\newcommand{\AttributeTok}[1]{\textcolor[rgb]{0.13,0.29,0.53}{#1}}
\newcommand{\BaseNTok}[1]{\textcolor[rgb]{0.00,0.00,0.81}{#1}}
\newcommand{\BuiltInTok}[1]{#1}
\newcommand{\CharTok}[1]{\textcolor[rgb]{0.31,0.60,0.02}{#1}}
\newcommand{\CommentTok}[1]{\textcolor[rgb]{0.56,0.35,0.01}{\textit{#1}}}
\newcommand{\CommentVarTok}[1]{\textcolor[rgb]{0.56,0.35,0.01}{\textbf{\textit{#1}}}}
\newcommand{\ConstantTok}[1]{\textcolor[rgb]{0.56,0.35,0.01}{#1}}
\newcommand{\ControlFlowTok}[1]{\textcolor[rgb]{0.13,0.29,0.53}{\textbf{#1}}}
\newcommand{\DataTypeTok}[1]{\textcolor[rgb]{0.13,0.29,0.53}{#1}}
\newcommand{\DecValTok}[1]{\textcolor[rgb]{0.00,0.00,0.81}{#1}}
\newcommand{\DocumentationTok}[1]{\textcolor[rgb]{0.56,0.35,0.01}{\textbf{\textit{#1}}}}
\newcommand{\ErrorTok}[1]{\textcolor[rgb]{0.64,0.00,0.00}{\textbf{#1}}}
\newcommand{\ExtensionTok}[1]{#1}
\newcommand{\FloatTok}[1]{\textcolor[rgb]{0.00,0.00,0.81}{#1}}
\newcommand{\FunctionTok}[1]{\textcolor[rgb]{0.13,0.29,0.53}{\textbf{#1}}}
\newcommand{\ImportTok}[1]{#1}
\newcommand{\InformationTok}[1]{\textcolor[rgb]{0.56,0.35,0.01}{\textbf{\textit{#1}}}}
\newcommand{\KeywordTok}[1]{\textcolor[rgb]{0.13,0.29,0.53}{\textbf{#1}}}
\newcommand{\NormalTok}[1]{#1}
\newcommand{\OperatorTok}[1]{\textcolor[rgb]{0.81,0.36,0.00}{\textbf{#1}}}
\newcommand{\OtherTok}[1]{\textcolor[rgb]{0.56,0.35,0.01}{#1}}
\newcommand{\PreprocessorTok}[1]{\textcolor[rgb]{0.56,0.35,0.01}{\textit{#1}}}
\newcommand{\RegionMarkerTok}[1]{#1}
\newcommand{\SpecialCharTok}[1]{\textcolor[rgb]{0.81,0.36,0.00}{\textbf{#1}}}
\newcommand{\SpecialStringTok}[1]{\textcolor[rgb]{0.31,0.60,0.02}{#1}}
\newcommand{\StringTok}[1]{\textcolor[rgb]{0.31,0.60,0.02}{#1}}
\newcommand{\VariableTok}[1]{\textcolor[rgb]{0.00,0.00,0.00}{#1}}
\newcommand{\VerbatimStringTok}[1]{\textcolor[rgb]{0.31,0.60,0.02}{#1}}
\newcommand{\WarningTok}[1]{\textcolor[rgb]{0.56,0.35,0.01}{\textbf{\textit{#1}}}}
\usepackage{longtable,booktabs,array}
\usepackage{calc} % for calculating minipage widths
% Correct order of tables after \paragraph or \subparagraph
\usepackage{etoolbox}
\makeatletter
\patchcmd\longtable{\par}{\if@noskipsec\mbox{}\fi\par}{}{}
\makeatother
% Allow footnotes in longtable head/foot
\IfFileExists{footnotehyper.sty}{\usepackage{footnotehyper}}{\usepackage{footnote}}
\makesavenoteenv{longtable}
\usepackage{graphicx}
\makeatletter
\def\maxwidth{\ifdim\Gin@nat@width>\linewidth\linewidth\else\Gin@nat@width\fi}
\def\maxheight{\ifdim\Gin@nat@height>\textheight\textheight\else\Gin@nat@height\fi}
\makeatother
% Scale images if necessary, so that they will not overflow the page
% margins by default, and it is still possible to overwrite the defaults
% using explicit options in \includegraphics[width, height, ...]{}
\setkeys{Gin}{width=\maxwidth,height=\maxheight,keepaspectratio}
% Set default figure placement to htbp
\makeatletter
\def\fps@figure{htbp}
\makeatother
\setlength{\emergencystretch}{3em} % prevent overfull lines
\providecommand{\tightlist}{%
  \setlength{\itemsep}{0pt}\setlength{\parskip}{0pt}}
\setcounter{secnumdepth}{5}
% definitions for citeproc citations
\NewDocumentCommand\citeproctext{}{}
\NewDocumentCommand\citeproc{mm}{%
  \begingroup\def\citeproctext{#2}\cite{#1}\endgroup}
\makeatletter
 % allow citations to break across lines
 \let\@cite@ofmt\@firstofone
 % avoid brackets around text for \cite:
 \def\@biblabel#1{}
 \def\@cite#1#2{{#1\if@tempswa , #2\fi}}
\makeatother
\newlength{\cslhangindent}
\setlength{\cslhangindent}{1.5em}
\newlength{\csllabelwidth}
\setlength{\csllabelwidth}{3em}
\newenvironment{CSLReferences}[2] % #1 hanging-indent, #2 entry-spacing
 {\begin{list}{}{%
  \setlength{\itemindent}{0pt}
  \setlength{\leftmargin}{0pt}
  \setlength{\parsep}{0pt}
  % turn on hanging indent if param 1 is 1
  \ifodd #1
   \setlength{\leftmargin}{\cslhangindent}
   \setlength{\itemindent}{-1\cslhangindent}
  \fi
  % set entry spacing
  \setlength{\itemsep}{#2\baselineskip}}}
 {\end{list}}
\usepackage{calc}
\newcommand{\CSLBlock}[1]{\hfill\break#1\hfill\break}
\newcommand{\CSLLeftMargin}[1]{\parbox[t]{\csllabelwidth}{\strut#1\strut}}
\newcommand{\CSLRightInline}[1]{\parbox[t]{\linewidth - \csllabelwidth}{\strut#1\strut}}
\newcommand{\CSLIndent}[1]{\hspace{\cslhangindent}#1}
\usepackage{tcolorbox}
\usepackage{amssymb}
\usepackage{yfonts}
\usepackage{bm}

\newtcolorbox{greybox}{
  colback=white,
  colframe=blue,
  coltext=black,
  boxsep=5pt,
  arc=4pt}
  
\newcommand{\ds}[4]{\sum_{{#1}=1}^{#3}\sum_{{#2}=1}^{#4}}
\newcommand{\us}[3]{\mathop{\sum\sum}_{1\leq{#2}<{#1}\leq{#3}}}

\newcommand{\ol}[1]{\overline{#1}}
\newcommand{\ul}[1]{\underline{#1}}

\newcommand{\amin}[1]{\mathop{\text{argmin}}_{#1}}
\newcommand{\amax}[1]{\mathop{\text{argmax}}_{#1}}

\newcommand{\ci}{\perp\!\!\!\perp}

\newcommand{\mc}[1]{\mathcal{#1}}
\newcommand{\mb}[1]{\mathbb{#1}}
\newcommand{\mf}[1]{\mathfrak{#1}}

\newcommand{\eps}{\epsilon}
\newcommand{\lbd}{\lambda}
\newcommand{\alp}{\alpha}
\newcommand{\df}{=:}
\newcommand{\am}[1]{\mathop{\text{argmin}}_{#1}}
\newcommand{\ls}[2]{\mathop{\sum\sum}_{#1}^{#2}}
\newcommand{\ijs}{\mathop{\sum\sum}_{1\leq i<j\leq n}}
\newcommand{\jis}{\mathop{\sum\sum}_{1\leq j<i\leq n}}
\newcommand{\sij}{\sum_{i=1}^n\sum_{j=1}^n}
	
\ifLuaTeX
  \usepackage{selnolig}  % disable illegal ligatures
\fi
\IfFileExists{bookmark.sty}{\usepackage{bookmark}}{\usepackage{hyperref}}
\IfFileExists{xurl.sty}{\usepackage{xurl}}{} % add URL line breaks if available
\urlstyle{same}
\hypersetup{
  pdftitle={Notes on the C Version of Smacof},
  pdfauthor={Jan de Leeuw - University of California Los Angeles},
  hidelinks,
  pdfcreator={LaTeX via pandoc}}

\title{Notes on the C Version of Smacof}
\author{Jan de Leeuw - University of California Los Angeles}
\date{Started October 10 2023, Version of October 30, 2023}

\begin{document}
\maketitle
\begin{abstract}
TBD
\end{abstract}

{
\setcounter{tocdepth}{4}
\tableofcontents
}
\textbf{Note:} This is a working paper which will be expanded/updated frequently. All suggestions for improvement are welcome. All Rmd, tex, html, pdf, R, and C files are in the public domain. Attribution
will be appreciated, but is not required. The files can be found at
\url{https://github.com/deleeuw/mdsStruct}.

\section{Introduction}\label{introduction}

The loss function in (metric, least squares, Euclidean, symmetric) Multidimensional Scaling (MDS)
is
\[
\sigma(X):=\frac12\jis w_{ij}(\delta_{ij}-d_{ij}(X))^2.
\]

This assumes symmetry and it uses all elements below the diagonal of both \(W\), \(\Delta\), and \(D(X)\).
For missing data we set \(w_{ij}=0\).

\section{Example}\label{example}

Here is a small input example.

\section{smacofStructure}\label{smacofstructure}

We use this example as input for the R version of \emph{smacofSort()}, which results in the
\emph{smacofStructure}

The \emph{dist} element in the data frame is zero, because we have not computed distances yet.
Given a configuration \(X\) the \emph{row} and \emph{col} elements in the data frame allow from
straightforward computation of distances. In the case of nonmetric MDS, or more generally
in MDS with transformed dissimilarities, the \emph{dhat} column will be filled as well.
The fact that preprocessing with \emph{smacofSort()} gives the ordered dissimilarities, as
well as the tie blocks, is especially useful in the ordinal case, both with monotone
polynomials and monotone splines. In the metric (ratio) case there is no need for the
\emph{dhat} column.

It is obvious how this \emph{smacofStructure} can be adapted if there are multiway data.If we have row-conditional or matrix-conditional data, then we use one of these smacofStructures for each
row or each matrix.

\section{Appendix: Code}\label{appendix-code}

We use \emph{qsort} to sort the rows of the input data frame by increasing delta. The sorting
is done in C, the R version is a wrapper around the compiled C code. The C code also
contains a main which analyzes the same small example as we have used in the text.
Compile with ``clang -o runner -O2 smacofSort.c'' and then start ``runner'' in the shell.
The C code uses the .C() interface in R, which can be improved using .Call(),
but probably in this case with little gain.

\section{smacofHildreth}\label{smacofhildreth}

Hildreth (1957)

Consider the QP problem of minimizing \(f(x)=\frac12(x-y)'W(x-y)\)
over all \(x\in\mathbb{R}^n\) satisfying \(Ax\geq 0\), where \(A\) is \(m\times n\). Wlg we can assume \(a_j'Wa_j=1\). The Lagrangian is
\[
\mathcal{L}(x,\lambda)=\frac12(x-y)'W(x-y)-\lambda'Ax.
\]
\[
\max_{\lambda\geq 0}\mathcal{L}(x,\lambda)=\begin{cases}\frac12(x-y)'W(x-y)&\text{ if }Ax\geq 0,\\
+\infty&\text{ otherwise}.\end{cases}
\]
and thus
\[
\min_{x\in\mathbb{R}^n}\max_{\lambda\geq 0}\mathcal{L}(x,\lambda)=\min_{Ax\geq 0}\frac12(x-y)'W(x-y).
\]
By duality
\[
\min_{x\in\mathbb{R}^n}\max_{\lambda\geq 0}\mathcal{L}(x,\lambda)=\max_{\lambda\geq 0}\min_{x\in\mathbb{R}^n}\mathcal{L}(x,\lambda).
\]
The inner minimum over \(x\) is attained for
\[
x=y+W^{-1}A'\lambda,
\]
and is equal to
\[
\min_{x\in\mathbb{R}^n}\mathcal{L}(x,\lambda)=-\frac12\lambda'AW^{-1}A'\lambda-\lambda'Ay.
\]
Thus
\[
\max_{\lambda\geq 0}\min_{x\in\mathbb{R}^n}\mathcal{L}(x,\lambda)=
-\frac12\min_{\lambda\geq 0}\left\{(Wy+A'\lambda)'W^{-1}(Wy+A'\lambda)-y'Wy\right\}
\]
We minimize \(h(\lambda)=(Wy+A'\lambda)'W^{-1}(Wy+A'\lambda)\) with coordinate descent. Let \(\lambda_j(\eps)=\lambda+\eps e_j\). Then
\[
h(\lambda_j(\eps))=(Wy+A'\lambda+\eps a_j)'W^{-1}()=\eps^2a_j'W^{-1}a_j+2\eps a_j'W^{-1}(y+W^{-1}A'\lambda)+
\]
which must be minimized over \(\eps\geq-\lambda_j\). So the minimum is attained at
\[
\eps=-\frac{a_j'W^{-1}x}{a_j'W^{-1}a_j}
\]
with \(x=y+W^{-1}A'\lambda\) (cf \ldots), provided \ldots{} satisfies .. Otherwise \(\eps=-\lambda_j\). Now update both \(\lambda\) and \(x\),
and go to the next \(j\).

\section{smacofDykstra}\label{smacofdykstra}

\section{smacofJacobi}\label{smacofjacobi}

taken from De Leeuw (2017)

\section{smacofSort}\label{smacofsort}

\section{Code}\label{code}

\subsection{smacofSort.R}\label{smacofsort.r}

\subsection{smacofSort.c}\label{smacofsort.c}

\begin{Shaded}
\begin{Highlighting}[]
\PreprocessorTok{\#include }\ImportTok{"smacof.h"}

\DataTypeTok{int}\NormalTok{ smacofComparison}\OperatorTok{(}\DataTypeTok{const} \DataTypeTok{void} \OperatorTok{*}\NormalTok{px}\OperatorTok{,} \DataTypeTok{const} \DataTypeTok{void} \OperatorTok{*}\NormalTok{py}\OperatorTok{)} \OperatorTok{\{}
    \DataTypeTok{double}\NormalTok{ x }\OperatorTok{=} \OperatorTok{((}\KeywordTok{struct}\NormalTok{ fiveTuple }\OperatorTok{*)}\NormalTok{px}\OperatorTok{){-}\textgreater{}}\NormalTok{delta}\OperatorTok{;}
    \DataTypeTok{double}\NormalTok{ y }\OperatorTok{=} \OperatorTok{((}\KeywordTok{struct}\NormalTok{ fiveTuple }\OperatorTok{*)}\NormalTok{py}\OperatorTok{){-}\textgreater{}}\NormalTok{delta}\OperatorTok{;}
    \ControlFlowTok{return} \OperatorTok{(}\DataTypeTok{int}\OperatorTok{)}\NormalTok{copysign}\OperatorTok{(}\FloatTok{1.0}\OperatorTok{,}\NormalTok{ x }\OperatorTok{{-}}\NormalTok{ y}\OperatorTok{);}
\OperatorTok{\}}

\DataTypeTok{void}\NormalTok{ smacofSort}\OperatorTok{(}\DataTypeTok{double} \OperatorTok{*}\NormalTok{delta}\OperatorTok{,} \DataTypeTok{double} \OperatorTok{*}\NormalTok{weight}\OperatorTok{,} \DataTypeTok{int} \OperatorTok{*}\NormalTok{row}\OperatorTok{,} \DataTypeTok{int} \OperatorTok{*}\NormalTok{col}\OperatorTok{,} \DataTypeTok{int} \OperatorTok{*}\NormalTok{index}\OperatorTok{,}
                \DataTypeTok{const} \DataTypeTok{int} \OperatorTok{*}\NormalTok{ndata}\OperatorTok{)} \OperatorTok{\{}
    \DataTypeTok{int}\NormalTok{ n }\OperatorTok{=} \OperatorTok{*}\NormalTok{ndata}\OperatorTok{;}
    \KeywordTok{struct}\NormalTok{ fiveTuple }\OperatorTok{*}\NormalTok{xi }\OperatorTok{=}
        \OperatorTok{(}\KeywordTok{struct}\NormalTok{ fiveTuple }\OperatorTok{*)}\NormalTok{calloc}\OperatorTok{((}\DataTypeTok{size\_t}\OperatorTok{)}\NormalTok{n}\OperatorTok{,} \OperatorTok{(}\DataTypeTok{size\_t}\OperatorTok{)}\KeywordTok{sizeof}\OperatorTok{(}\KeywordTok{struct}\NormalTok{ fiveTuple}\OperatorTok{));}
    \ControlFlowTok{for} \OperatorTok{(}\DataTypeTok{int}\NormalTok{ i }\OperatorTok{=} \DecValTok{0}\OperatorTok{;}\NormalTok{ i }\OperatorTok{\textless{}}\NormalTok{ n}\OperatorTok{;}\NormalTok{ i}\OperatorTok{++)} \OperatorTok{\{}
\NormalTok{        xi}\OperatorTok{[}\NormalTok{i}\OperatorTok{].}\NormalTok{index }\OperatorTok{=}\NormalTok{ i}\OperatorTok{;}
\NormalTok{        xi}\OperatorTok{[}\NormalTok{i}\OperatorTok{].}\NormalTok{row }\OperatorTok{=}\NormalTok{ row}\OperatorTok{[}\NormalTok{i}\OperatorTok{];}
\NormalTok{        xi}\OperatorTok{[}\NormalTok{i}\OperatorTok{].}\NormalTok{col }\OperatorTok{=}\NormalTok{ col}\OperatorTok{[}\NormalTok{i}\OperatorTok{];}
\NormalTok{        xi}\OperatorTok{[}\NormalTok{i}\OperatorTok{].}\NormalTok{delta }\OperatorTok{=}\NormalTok{ delta}\OperatorTok{[}\NormalTok{i}\OperatorTok{];}
\NormalTok{        xi}\OperatorTok{[}\NormalTok{i}\OperatorTok{].}\NormalTok{weight }\OperatorTok{=}\NormalTok{ weight}\OperatorTok{[}\NormalTok{i}\OperatorTok{];}
    \OperatorTok{\}}
    \OperatorTok{(}\DataTypeTok{void}\OperatorTok{)}\NormalTok{qsort}\OperatorTok{(}\NormalTok{xi}\OperatorTok{,} \OperatorTok{(}\DataTypeTok{size\_t}\OperatorTok{)}\NormalTok{n}\OperatorTok{,} \OperatorTok{(}\DataTypeTok{size\_t}\OperatorTok{)}\KeywordTok{sizeof}\OperatorTok{(}\KeywordTok{struct}\NormalTok{ fiveTuple}\OperatorTok{),}
\NormalTok{                smacofComparison}\OperatorTok{);}
    \ControlFlowTok{for} \OperatorTok{(}\DataTypeTok{int}\NormalTok{ i }\OperatorTok{=} \DecValTok{0}\OperatorTok{;}\NormalTok{ i }\OperatorTok{\textless{}}\NormalTok{ n}\OperatorTok{;}\NormalTok{ i}\OperatorTok{++)} \OperatorTok{\{}
\NormalTok{        index}\OperatorTok{[}\NormalTok{i}\OperatorTok{]} \OperatorTok{=}\NormalTok{ xi}\OperatorTok{[}\NormalTok{i}\OperatorTok{].}\NormalTok{index}\OperatorTok{;}
\NormalTok{        row}\OperatorTok{[}\NormalTok{i}\OperatorTok{]} \OperatorTok{=}\NormalTok{ xi}\OperatorTok{[}\NormalTok{i}\OperatorTok{].}\NormalTok{row}\OperatorTok{;}
\NormalTok{        col}\OperatorTok{[}\NormalTok{i}\OperatorTok{]} \OperatorTok{=}\NormalTok{ xi}\OperatorTok{[}\NormalTok{i}\OperatorTok{].}\NormalTok{col}\OperatorTok{;}
\NormalTok{        delta}\OperatorTok{[}\NormalTok{i}\OperatorTok{]} \OperatorTok{=}\NormalTok{ xi}\OperatorTok{[}\NormalTok{i}\OperatorTok{].}\NormalTok{delta}\OperatorTok{;}
\NormalTok{        weight}\OperatorTok{[}\NormalTok{i}\OperatorTok{]} \OperatorTok{=}\NormalTok{ xi}\OperatorTok{[}\NormalTok{i}\OperatorTok{].}\NormalTok{weight}\OperatorTok{;}
    \OperatorTok{\}}
\NormalTok{    free}\OperatorTok{(}\NormalTok{xi}\OperatorTok{);}
    \ControlFlowTok{return}\OperatorTok{;}
\OperatorTok{\}}

\DataTypeTok{void}\NormalTok{ smacofTieBlocks}\OperatorTok{(}\DataTypeTok{const} \DataTypeTok{double} \OperatorTok{*}\NormalTok{delta}\OperatorTok{,} \DataTypeTok{int} \OperatorTok{*}\NormalTok{block}\OperatorTok{,} \DataTypeTok{double} \OperatorTok{*}\NormalTok{eps}\OperatorTok{,}
                     \DataTypeTok{const} \DataTypeTok{int} \OperatorTok{*}\NormalTok{ndata}\OperatorTok{)} \OperatorTok{\{}
    \DataTypeTok{int}\NormalTok{ n }\OperatorTok{=} \OperatorTok{*}\NormalTok{ndata}\OperatorTok{;}
\NormalTok{    block}\OperatorTok{[}\DecValTok{0}\OperatorTok{]} \OperatorTok{=} \DecValTok{1}\OperatorTok{;}
    \ControlFlowTok{for} \OperatorTok{(}\DataTypeTok{int}\NormalTok{ i }\OperatorTok{=} \DecValTok{1}\OperatorTok{;}\NormalTok{ i }\OperatorTok{\textless{}}\NormalTok{ n}\OperatorTok{;}\NormalTok{ i}\OperatorTok{++)} \OperatorTok{\{}
        \ControlFlowTok{if} \OperatorTok{(}\NormalTok{fabs}\OperatorTok{(}\NormalTok{delta}\OperatorTok{[}\NormalTok{i}\OperatorTok{]} \OperatorTok{{-}}\NormalTok{ delta}\OperatorTok{[}\NormalTok{i }\OperatorTok{{-}} \DecValTok{1}\OperatorTok{])} \OperatorTok{\textless{}} \OperatorTok{*}\NormalTok{eps}\OperatorTok{)} \OperatorTok{\{}
\NormalTok{            block}\OperatorTok{[}\NormalTok{i}\OperatorTok{]} \OperatorTok{=}\NormalTok{ block}\OperatorTok{[}\NormalTok{i }\OperatorTok{{-}} \DecValTok{1}\OperatorTok{];}
        \OperatorTok{\}} \ControlFlowTok{else} \OperatorTok{\{}
\NormalTok{            block}\OperatorTok{[}\NormalTok{i}\OperatorTok{]} \OperatorTok{=}\NormalTok{ block}\OperatorTok{[}\NormalTok{i }\OperatorTok{{-}} \DecValTok{1}\OperatorTok{]} \OperatorTok{+} \DecValTok{1}\OperatorTok{;}
        \OperatorTok{\}}
    \OperatorTok{\}}
    \ControlFlowTok{return}\OperatorTok{;}
\OperatorTok{\}}
\end{Highlighting}
\end{Shaded}

\section*{References}\label{references}
\addcontentsline{toc}{section}{References}

\phantomsection\label{refs}
\begin{CSLReferences}{1}{0}
\bibitem[\citeproctext]{ref-deleeuw_E_17o}
De Leeuw, J. 2017. {``{Jacobi Eigen in R/C with Lower Triangular Column-wise Compact Storage}.''} 2017.

\bibitem[\citeproctext]{ref-hildreth_57}
Hildreth, C. 1957. {``{A Quadratic Programming Procedure}.''} \emph{Naval Research Logistic Quarterly} 14 (79--85).

\end{CSLReferences}

\end{document}
